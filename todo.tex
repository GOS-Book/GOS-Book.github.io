todolist

 2) жирные просьбы - теорема о монотонных односторонних, сократить параграф с обозн., посмотреть используется конечная-произвбеск.
\3)textbf{CМОТРИ В БИЛЕТЕ 6 пока что. + странно что материал страниц 116 нигде не пригодился, хотя это фундаментальные понятия и леммы}
 4)картиночки
 5)добавить везде ВСЕ нужные теоремы и определения
 6)если успеется. то и все-все доказательства
 7)орнамент?)
 8)рисунок райгородского в формуле полной вероятности?
 9) надо оформить все функции вставляющие картинки как \usepict с автоматическим label и pictures/
 
 10) надо сделать рисунок катета и гипотенузы в УКР билета 33
 11) Из теоремы \hyperref[exp14]{о трех функциях} следует: $\exists \lim_{\substack{\Delta x \to 0\\ \Delta y \to 0}}\limits \frac{|\alpha_1(\Delta x,\Delta y)|}{\sqrt{\Delta x ^2 + \Delta y^2}} = 0.$
Аналогично, $\exists \lim_{\substack{\Delta x \to 0\\ \Delta y \to 0}}\limits \frac{|\alpha_2(\Delta x,\Delta y)|}{\sqrt{\Delta x^2 + \Delta y^2}} = 0$. 
Отсюда равенства \eqref{exp15} означают дифференцируемость функций $u(x,y)$ и $v(x,y)$ в точке $(x_0,y_0) \in \bbR^2$, причем 


33 билет - некрасиво же.

12) картинка ,33.4 там Г хотя я в книге использую обозначение партиал Ж для края

13) относительно 12 подумай какое лучше обозначение для края будет с плюсиком или без плюсика. 

в header.tex есть строчка геометрии. перед каждым аутпутом надо ее менять на ровные границы.