\chapter{Предисловие}
Здравствуй, мой дорогой читатель. Написать данное учебное пособие для меня в одиночку --- было титаническим трудом. Но я думаю, что справился достойно.

Данное учебное пособие предназначено для подготовки конкретно к устному ГОСу по математике. Как вы уже поняли, оглавление представляет собой программу к ГОСУ 2016 года. Заметьте, что она <<кликабельна>>, так что умелые люди при должной сноровке смогут быстро на экзамене найти почти любой материал, который касается именно программы ГОСа. И <<кликабельно>> не только оглавление данной книги, но и всевозможные числа и названия, указывающие на теоремы, которые использовались ранее. Надеюсь, это кому-нибудь поможет. 

Надо бы отметить для любителей учебников определенных авторов, что я писал билеты, существенно опираясь на учебные пособия, лекции различных преподавателей. Вот список соответствия билетов и материала, которыми я в основном пользовался:
\begin{itemize}
\item[1-4]
\; --- \: Лекции Сакбаева В.Ж. и учебное пособие Яковлева Г.Н.
\item[33-36]
\; --- \:
\item[\sffamily 33-36]
\; --- \:
\item[\sffamily 33-36]
\; --- \:
\item[\sffamily 33-36]
\; --- \:
\item[\sffamily 33-36]
\; --- \:
\item[33-36]
\; --- \: Лекции Карлова М.И. и учебное пособие Половинкина Е.С.
\end{itemize}

Я был предельно внимателен к составлению данной книги, стараясь уменьшить количество опечаток и повысить качество излагаемого материала. Но я отказываюсь от ответственности за всевозможные недочеты в этой книге, в конце концов я работал почти в одиночку, и я, на данный момент, всего лишь студент 3-ого курса. И поэтому, прошу Вас не забывать отправить мне (\url{https://vk.com/didenko.andre}) сообщения о любых неточностях, опечатках, ошибках (даже о таких, как, например, <<забыл точку поставить>>). Также пишите, если хотите дать совет или выразить личные пожелания. 
 
\vspace*{\baselineskip}
Желаю всем отличных результатов на ГОСе.