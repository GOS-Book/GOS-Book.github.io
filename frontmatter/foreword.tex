\chapter*{Предисловие}
\addcontentsline{toc}{chapter}{Предисловие}

\begin{center} 
Здравствуй, мой дорогой читатель!
\end{center}

Пожалуй, стоит начать разговор с того, откуда появилась идея написать эту книгу. Январь 2016 года, 3-ий курс, скоро ГОС-экзамен по физике, в общежитии на полке у каждого студента 10--20 книг по физике (а то и больше у некоторых!), которые мы тащили из библиотеки себе и девочкам: толстенький том Сивухина за каждый курс; небольшие, но емкие книжки Кириченко; кто-то Кингсепа любит; еще дополнительная литература всякая и так далее... Кто не любит бумажные книжки "--- мучается с бесконечным количеством pdf-файлов на ноутбуке. ``И это не круто!'' "--- посчитал я, неудобно это возиться с таким большим количеством книг, идеально будет иметь лишь одну книжку со всеми ответами, со всеми билетами, но которая все равно старается охватить весь материал на фундаментальном уровне. Предложил друзьям начать \textit{ботать} (любимое слово физтеха) ГОС-экзамен по математике, начиная с самого февраля и стараться техать все вопросы, чтобы совместно все \textit{зашарить} (другое любимое слово физтеха). Так и началась кропотливая работа нашей небольшой редакции.


И как итог этой небольшой истории: пособие, которое ты сейчас читаешь, "--- результат титанического труда многих людей. И мы все верим, что результат получился более чем достойный. 

Цель данной книги "--- облегчить подготовку студентов Московского Физико"--~Технического Института к устному выпускному квалификационному государственному экзамену по математике, попросту к ГОСу. Как вы уже поняли, оглавление представляет собой программу к ГОСу (версию программы~2018 года, если быть точным). Заметьте, что оно кликабельно в pdf-версии данной книги, что упрощает работу с книгой. И кликабельно не только оглавление данной книги, но всевозможные числа и названия, указывающие на теоремы, которые уже использовались ранее в книге, а также объекты в мини-содержаниях перед главами. Старались максимально использовать достоинства верстки на \LaTeX. Надеюсь, это кому-нибудь поможет. 

Нужно отметить, что мы писали материал глав, существенно опираясь на учебные пособия, лекции различных преподавателей. Список соответствия билетов из программы и названий курсов от кафедры высшей математики материалам, которыми мы в основном пользовались смотрите ранее, в списке литературы.

Мы были предельно внимательны к составлению данной книги, стараясь уменьшить количество опечаток и повысить качество излагаемого материала. Но мы отказываемся от ответственности за всевозможные недочеты в этой книге (пожалуй, главный недочет этой книги "--- чрезмерная избыточность в некоторых местах, а иногда, наоборот, недостаток материала), ведь мы, на данный момент, всего лишь студенты, а главное "--- люди, которые могут ошибаться. И поэтому, прошу вас не забывать отправлять нам (ссылки на титульной странице) сообщения о любых неточностях, опечатках, ошибках, недочетах. Также пишите, если хотите дать совет или выразить любые личные пожелания. Вместе с вами можно довести эту книгу до очень хорошего пособия. И поверьте, каждое сообщение (даже о самой незначительной опечатке) очень важно и их приятно читать и получать.

Необходимо сразу предостеречь читателя, что данное пособие является именно кратким пересказом курса математики на физтехе, охватывающим вопросы к ГОСу. Пересказ должен быть цельным, а не отрывочным "--- как предлагает программа ГОСа, поэтому материал в билетах порой является избыточным для рассказа билета на самом экзамене. Например, я бы не стал сам в первом билете говорить об аксиоматике множества действительных чисел, однако поместить ее в это место в книге я просто обязан для полной картины мира. Однако оставляю читателю полную свободу выбора тем разговора с преподавателем. Названия глав "--- это лишь названия билетов из программы, контрольные точки, к которым плавно материал книги пытается подвести читателя.

Не могу не отметить доброжелательного и внимательного отношения всех студентов МФТИ к этому пособию. Хочу сказать всем, кто присылал сообщения об опечатках и ошибках: ``Спасибо''. Также хочется выразить особую благодарность Кудашову Аркадию, Лузянину Артемию, Проскину Роману, Вербе Глебу и Браславскому Илье за непосредственное соучастие в написании этой книги и выразить признательность Брицыну Евгению и Дроботу Олегу за многочисленные комментарии и исправления.

Не обошлось даже без участия преподавательского состава МФТИ. Так, например, Максим Широбоков, преподаватель теории вероятностей, случайных процессов и математической статистики, прочитал главы по теории вероятностей, оставил важные и ценные комментарии, и в ходе долгой дискуссии после редактирования глав одобрил последние. За это все я очень благодарен ему.

Также Чубаров Игорь Андреевич на своей очной консультации упомянул, что читал это пособие, и сказал, что главы по его предмету (аналитическая геометрия и линейная алгебра) написаны хорошо.
 
Мне лишь остается выразить надежду, что настоящее пособие поможет студентам при изучении математики в целом и подготовке к ГОСу. И жду ваших отзывов и помощи, потому что признаю, что с задачей, которую я себе ставил "--- написать идеальное пособие, которое покрывает весь материал ГОСа по математике физтеха на абсолютно \textit{фундаментальном} уровне "--- я не справился.

\textcolor{red}{
$\blacktriangleright$
И именно по этой причине я крайне не рекомендую использовать данное пособие как \textit{\textbf{основную}} литературу во время подготовки к Государственному экзамену по математике.
}

\vspace*{0.7\baselineskip} 

\textit{Желаю всем отличных результатов~на~ГОСе.}

\mbox{}

\noindent\textit{Диденко Андрей}