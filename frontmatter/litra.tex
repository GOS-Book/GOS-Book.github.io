\chapter{Список Литературы}

Предоставляю список литературы, которыми мы пользовались в основном для написания билетов. 

\begin{itemize}
\item[\textit{1-6}]
\; --- \: \textit{Введение в математический анализ.} 

Лекции Сакбаева В.Ж. и учебное пособие Яковлева Г.Н. (часть 1), учебное пособие Бесова О.В.
\item[\textit{7-8}] 
\; --- \: \textit{Многомерный анализ, интегралы и ряды.}

Лекции
\item[\textit{9-10}] 
\; --- \: \textit{Кратные интегралы и теория поля.}

Лекции
\item[\textit{11-13}] 
\; --- \: \textit{Многомерный анализ, интегралы и ряды.}

Лекции
\item[\textit{14-16}] 
\; --- \: \textit{Кратные интегралы и теория поля.}

Лекции
\item[\textit{17-19}] 
\; --- \: \textit{Гармонический анализ.}

Лекции Сакбаева В.Ж. и учебное пособие Яковлева Г.Н. (часть 3).
\item[\textit{20}] 
\; --- \: \textit{Аналитическая геометрия.}

Лекции Чубарова И.А. и учебное пособие Беклемишева Д.В.
\item[\textit{21-25}] 
\; --- \: \textit{Линейная алгебра.}

Лекции Чубарова И.А. и учебное пособие Беклемишева Д.В.
\item[\textit{26-29}] 
\; --- \: \textit{Дифференциальные уравнения.}

Учебное пособие Романко В.К.

\item[\textit{30-32}]
\; --- \: \textit{Теория вероятностей.}

Лекции Райгородского А.М., семинарские записи Карлова М.И., учебное пособие Гнеденко Б.В., учебное пособие трех авторов: Захарова В.К., Севастьянов Б.А., Чистякова В.П.
\item[\textit{33-36}]
\; --- \: \textit{Теория функций комплексного переменного.}

Лекции Карлова М.И. и учебное пособие Половинкина Е.С.
\end{itemize}
