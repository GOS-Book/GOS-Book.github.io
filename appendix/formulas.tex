\chapter{Формулы к письменному ГОСу}

\titlespacing{\subsubsection}{0pt}{0mm plus 1mm minus 1mm}{0pt plus 1mm minus 1mm}  

\section[Тригонометрические формулы.]{Тригонометрические формулы.\footnote{Здесь не указаны допустимые значения аргумента в виду их очевидности (знаменатели не ноль).}}
\subsubsection{Основные тригонометрические формулы.}
\vspace*{-1.8\baselineskip}
\begin{multicols}{3}
\begin{align*}
&\sin^2x+\cos^2x=1;\\
\end{align*}
\vfill
\columnbreak
\begin{align*}
&\tg^2x+1=\frac{1}{\cos^2x};\\
&\ctg^2x+1=\frac{1}{\sin^2x};
\end{align*}
\vfill
\columnbreak
\begin{align*}
&\tg x\cdot\ctg x=1.\\
\end{align*}
\vfill
\end{multicols}
\subsubsection{Формулы сложения и вычитания аргументов.}
\vspace*{-1.8\baselineskip}
\begin{multicols}{2}
\begin{align*}
&\sin(x\pm y)=\sin x \cos y \pm \cos x \sin y;\\
&\cos(x\pm y)= \cos x \cos y \mp \sin x \sin y;\\
\end{align*}
\vfill
\columnbreak
\begin{align*}
&\tg(x\pm y)=\frac{\tg x \pm \tg y}{1 \mp \tg x\tg y};\\
&\ctg(x\pm y)=\frac{\ctg x\ctg y \mp 1}{\ctg x \pm \ctg y}.
\end{align*}
\end{multicols}
\subsubsection{Формулы двойного угла.}
\vspace*{-1.8\baselineskip}
\begin{multicols}{3}
\begin{align*}
&\sin 2x = 2\sin x \cos x;\\ 
\end{align*}
\vfill
\columnbreak
\begin{align*}
&\cos 2x = \cos^2 x - \sin^2 x;\\
&\cos 2x = 2\cos^2 x - 1; \\
&\cos 2x = 1 - 2\sin^2 x;
\end{align*}
\vfill
\columnbreak
\begin{align*}
&\tg 2x = \frac{2\tg x}{1-\tg^2x};\\
&\ctg 2x = \frac{\ctg^2x-1}{2\ctg x}.
\end{align*}
\end{multicols}
\subsubsection{Формулы понижения степени.}
\vspace*{-1.8\baselineskip}
\begin{multicols}{2}
\begin{align*}
&2\sin^2x=1-\cos 2x;\\
&4\sin^3x=3\sin x-\sin 3x;\\
&8\sin^4x=3-4\cos2x+\cos4x;\\
&16\sin^5x=10\sin x-5\sin 3x +\sin 5x;
\end{align*}
\vfill
\columnbreak
\begin{align*}
&2\cos^2x=1+\cos 2x;\\
&4\cos^3x=3\cos x+\cos 3x;\\
&8\cos^4x=3+4\cos2x+\cos4x;\\
&16\cos^5x=10\cos x+5\cos 3x +\cos 5x.
\end{align*}
\end{multicols}
\subsubsection{Формулы преобразования произведений функций.}
\vspace*{-0.8\baselineskip}
\begin{align*}
&2\sin x \sin y = \cos (x-y)-\cos(x+y);\\
&2\cos x \cos y = \cos (x-y)+\cos(x+y);\\
&2\sin x \cos y = \sin (x-y)+\sin(x+y);
\end{align*}
\newpage
\subsubsection{Формулы преобразования суммы функций.}
\vspace*{-1.8\baselineskip}
\begin{multicols}{2}
\begin{align*}
&\sin x\pm \sin y =2 \sin \frac{x\pm y}{2} \cos \frac{x\mp y}{2};\\
&\cos x + \cos y  = 2 \cos \frac{x + y}{2} \cos \frac{x - y}{2};\\
&\cos x - \cos y  = -2 \sin \frac{x + y}{2} \sin\frac{x - y}{2};
\end{align*}
\vfill
\columnbreak
\begin{align*}
&\tg x \pm \tg y = \frac{\sin (x \pm y)}{ \cos x \cos y};\\
&\ctg x \pm \tg y = \frac{\sin (y \pm x)}{ \sin x \sin y}.
\end{align*}
\end{multicols}

\subsubsection{Формула Эйлера и ее следствия.}
$e^{ix} = \cos x + i \sin x$

$\sin x = \frac{e^{ix}-e^{-ix}}{2i}$

$\cos x = \frac{e^{ix}+e^{-ix}}{2}$

\section{Гиперболические функции.}

\subsubsection{Определение.}
$\sh x = \frac{e^{x}-e^{-x}}{2};$

$\ch x = \frac{e^{x}+e^{-x}}{2};$

$\th x = \frac{\sh x}{\ch x}=\frac{e^{x}-e^{-x}}{e^{x}+e^{-x}}=\frac{e^{2x}-1}{e^{2x}+1};$

$\cth x = \frac{1}{\th x};$

\subsubsection{Связь с тригонометрическими функциями.}

$\ch (ix) = \cos x$

$\cos (ix) = \ch x$

$\sh (ix)= i \sin x$

$\sin (ix) = i \sh x$

$\th (ix) = i \tg x$

$\tg (ix) = i \th x$

\subsubsection{Важное соотношение.}
$\ch^2 x - \sh^2 x = 1;$

\subsubsection{Четность.}

\subsubsection{Формулы сложения.}

\subsubsection{Формулы двойного угла.}

\subsubsection{Формулы кратных углов.}

\subsubsection{Формулы произведения.}

\subsubsection{Формулы суммы.}

\subsubsection{Формулы понижения степени.}

\section{Обратные гиперболические функции.}


\section[Ряд Тейлора.]{Ряд Тейлора.\footnote{При $x \to 0$, R --- радиус сходимости.}}
\vspace*{-1\baselineskip}
\renewcommand*{\arraystretch}{2}
\noindent\begin{tabular}{ l l l }
\textbullet
&
$e^x=1+\frac{x}{1!}+\frac{x^2}{2!}+\frac{x^3}{3!}+o(x^3);$
&
$e^x=\sum\limits_{n=0}^{+\infty}\displaystyle\frac{x^n}{n!},\ R=\infty;$
\\
\textbullet
&
$\sin x = x - \frac{x^3}{3!} + \frac{x^5}{5!}+o(x^5);$
&
$\sin x = \sum\limits_{n=0}^{+\infty} (-1)^{n}\displaystyle\frac{x^{2n+1}}{(2n+1)!},\ R=\infty$
\\
&
$\cos x = 1 - \frac{x^2}{2!} + \frac{x^4}{4!}+o(x^4);$
&
$\cos x  = \sum\limits_{n=0}^{+\infty} (-1)^{n}\displaystyle\frac{x^{2n}}{(2n)!}$
\\
&
$\tg x = x + \frac{x^3}{3}+\frac{2x^5}{15}+o(x^5)$
&
$\tg x  = \sum\limits_{n=0}^{+\infty}$
\\
\textbullet
&
$\sh x = x +\frac{x^3}{3!}+\frac{x^5}{5!}+o(x^5);$
&
$\sh x =\sum\limits_{n=0}^{+\infty}\displaystyle\frac{x^{2n+1}}{(2n+1)!},\ R=\infty;$
\\
&
$\ch x = 1 + \frac{x^2}{2!}+\frac{x^4}{4!}+o(x^4);$
&
$\ch x  = \sum\limits_{n=0}^{+\infty}\displaystyle\frac{x^{2n}}{(2n)!},\ R=\infty;$
\\
&
$\th x = x - \frac{x^3}{3}+\frac{2x^5}{15}+o(x^5);$
&
$\th x  = \sum\limits_{n=0}^{+\infty};$
\\
&
$\cth x = \frac{1}{x} + \frac{x}{3}+\frac{x^3}{45}+o(x^3);$
&
$\cth x  = \sum\limits_{n=0}^{+\infty} B_{2n};$
\\
\textbullet
&
$\arcsin x = x + \frac{x^3}{6}+\frac{3x^5}{40}+o(x^5);$
&
$\arcsin x=\sum\limits_{n=0}^{+\infty};$
\\
&
$\arccos x= \frac{\pi}{2} - \arcsin x;$
&
$\arccos x = \sum\limits_{n=0}^{+\infty};$
\\
&
$\arctg x = x - \frac{x^3}{3}+\frac{x^5}{5}+o(x^5);$
&
$\arctg x = \sum\limits_{n=0}^{+\infty} (-1)^n\displaystyle\frac{x^{2n+1}}{2n+1};$
\\
\textbullet
&
$\arsh x = x - \frac{x^3}{6}+\frac{3x^5}{40}+o(x^5);$
&
$\arsh x = \sum\limits_{n=0}^{+\infty};$
\\
&
$\arch x = $
&
$\arch x = \sum\limits_{n=0}^{+\infty};$
\\
&
$\arth x = x + \frac{x^3}{3}+\frac{x^5}{5}+o(x^5);$
&
$\arth x = \sum\limits_{n=0}^{+\infty};$
\\
\textbullet
&
$\ln(1+x)=x-\frac{x^2}{2}+\frac{x^3}{3}+o(x^3);$
&
$\ln(1+x)= \sum\limits_{n=1}^{+\infty} \frac{(-1)^{n+1}}{n} x^n, R=1;$
\\
\textbullet
&
$
(1+x)^{a}=1+ax+\displaystyle\frac{a(a-1)}{2!}x^2+
$
&
$(1+x)^{a}= \sum\limits_{n=0}^{+\infty} C^{n}_{a} x^n,\;\text{где} $
\\
&
$\qquad+\displaystyle\frac{a(a-1)(a-2)}{3!}x^3+o(x^3);$
&
$\quad C^{n}_{a}=\frac{a(a-1)\dots(a-n+1)}{n!},\; R=1;$
\\
&
$\frac{1}{1-x}=1+x+x^2+x^3+o(x^3);$
&
$\frac{1}{1-x}= \sum\limits_{n=0}^{+\infty} x^{n};$
\end{tabular}

\section{Производные.}
\renewcommand*{\arraystretch}{2}
\noindent\begin{tabular}{ l l }
$(C)'=0$ ($C$ --- постоянная); 
&
$(\arcsin x)' = \displaystyle\frac{1}{\sqrt{1-x^2}};$
\\
$(x^n)' = nx^{n-1}, n\in \bbN;$
&
$(\arccos x)' = -\displaystyle\frac{1}{\sqrt{1-x^2}};$
\\
$(a^x)'=\ln a\cdot a^x,\ a>0;$
&
$(\arctg x)' = \displaystyle\frac{1}{1+x^2}$
\\
$(x^a)'=ax^{a-1},\ x>0;$
&
$(\arcctg x)' = -\displaystyle\frac{1}{1+x^2}$
\\
$(\log_a x)'=\displaystyle\frac{\log_a e}{x},\ x>0,\ a>0, a\ne 0;$
&
$(\sh x)'= \ch x;$
\\
$(\ln x)' = \displaystyle\frac{1}{x},\ x>0;$
&
$(\ch x)'= \sh x;$
\\
$(\ln |x|)' = \displaystyle\frac{1}{x},\ x\ne 0;$
&
$(\th x)'= \displaystyle\frac{1}{\ch^2 x};$
\\
$(\sin x)'=\cos x;$
&
$(\cth x)'= -\displaystyle\frac{1}{\sh^2 x};$
\\
$(\cos x)'=-\sin x;$
&
\\
$(\tg x)'= \displaystyle\frac{1}{\cos^2 x}, x \ne \frac{\pi}{2}(2n+1),\, n\in \bbZ;$
&
\\
$(\ctg x)'= -\displaystyle\frac{1}{\sin^2 x};$
&
$(e^x)'=e^x;$
\\
\end{tabular}

\section{Неопределенные Интегралы.}
$\int \frac{dx}{\sqrt{a^2-x^2}}=\arcsin\frac{x}{a}+C,\ a\ne 0$

\section{Вычисление площадей плоских фигур и длин кривых.}

\section{Вычисление объемов тел и площадей поверхностей.}

\section{Несобственные интегралы.}

$\displaystyle\int\limits_{0}^{1} \frac{dx}{x^a}$

\noindent$a<1\ \text{--- cх-ся};\\a\ge1\  \text{--- раcх-ся};$

\section{Признаки Даламбера и Коши.}

\section{3.9}

\section{Криволинейные интегралы.}
\subsubsection{первого}
$\int\limits_{\Gamma} F\, ds = \int\limits_{\alpha}^{\beta} F(x(t),y(t),z(t))\sqrt{(x'(t))^2+(y'(t))^2+(z'(t))^2}\,dt$
\subsubsection{второго}
$\int\limits_{\Gamma} \left(P\,dx+Q\,dy+R\,dz=\int\limits_{\alpha}^{\beta} P x'(t)+ Q y'(t)+ R z'(t)\right)\,dt$
\subsubsection{грина}


\section{Поверхностные интегралы.}

\section{Теория поля.}

\section{Фурье.}