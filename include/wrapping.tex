%%%%%%%%%%%%%%%%%%%%%%%%%%%%%%%%%%%%%%%%%%%%%%%%%%%%%%%%%%%%%%%%%%%
%%%%%%%%%%%%%%%%%%             RULES             %%%%%%%%%%%%%%%%%%
%%%%%%%%%%%%%%%%%%%%%%%%%%%%%%%%%%%%%%%%%%%%%%%%%%%%%%%%%%%%%%%%%%% 

% \usepict : To use it outside of everything, the simliest construction, not even wrapped:  
% \usepict[the number of supplementary lines to be \vspace*{-#1\baselineskip}]{picturename without pictures/}{part of textHEIGHT}
% Example from chapter35, defintion16:
\iffalse
\usepict{ch35pict4}{0.15}
Доказательство достаточно провести для случая, когда $G$ есть звездное множество с центром в точке $z_0 = 0$. Покажем это. 
\fi

% \insertpicture : To use inside environments like proof:  
% \insertpicture[the number of supplementary lines to be shortened = 5 by default]{picturename without pictures/}{part of textWIDTH occupied by picture}
% TIPS 
% 1. Sometimes you need to use \newpage in order to make correct spacing
% Example from chapter9, inside proof:
\iffalse
\insertpicture{ch9pict1}{0.42}
Отсюда следует, что $F(x,a)<0<F(x,b)$ для любого $x$ из интервала $\Delta = \Delta'\cap\Delta''$. А так как функция $F(x,y)$ при каждом фиксированном $x\in\Delta$ по $y$ непрерывна и строго монотонна, то для каждого $x\in \Delta$ существует единственное $y$, которое обозначим $f(x)$, такое что $f(x)\in(a;b)$ и $F(x,f(x))=0$. Следовательно, на прямоугольнике $\Delta\times(a;b)$ уравнение $F(x,y)=0$ определяет единственную неявную функцию $y=f(x)$. Докажем, что она непрерывна в точке $x_0$.
\fi

% \proofpicture : To use right after the bery first line in proof envinronemnt.
% \proofpicture[the number of supplementary lines to be shortened = 5 by default]{picturename without pictures/}{part of textWIDTH occupied by picture}{the very first line of proof which needs to fit space between word Доказательство and end of the line}}
% Example from chapter35, theorem3:
\iffalse
 \begin{proof}
 \proofpicture{ch35pict2}{0.2}{Для  заданного в теореме контура $\gamma$ запишем форму-}
 лу~\eqref{exp24} в виде:
\fi

% \itempicture : To use in lists. You need to use it together with 
% \wrapitemenumi; \wrapitemi: item with wrapped image (two version for enumarate and itemize respectively)
% \contitem : the following items which should be also affected by image.
% \MoreInsert : the following numbers of text lines afterthe list the which should be also affected by image.
% \proofpicture[the number of supplementary lines to be shortened = 5 by default]{picturename without pictures/}{part of textWIDTH occupied by picture}
% TIPS
% Also dont forget to use \end{enumerate}\vspace*{-\baselineskip}\begin{enumerate}[resume*] between \itempicture's if you use several \itempicture in one list 
% use \EndInsert to reset margins when it doesnt do it automatically, for example when lists are in proof
% Example from chapter35, definiton14:
\iffalse
\begin{enumerate}
	\itempicture{ch35pict3}{0.3}
	Кривая $\Gamma_k$ "--- \textit{правильной гладкой компонентой} $\Gamma$, если в каждой окрестности каждой точки $z_0 \in \Gamma_k$ находятся как точки из области $G$, так и из $\bbC \setminus G\cup\Gamma$.
	\contitem Кривая $\Gamma_k$ "--- \textit{разрезом}, если для каждой точки $z_0 \in\buildrel\circ\over\Gamma_k$ (обозначение $\buildrel\circ\over\Gamma_k$ "--- кривая $\Gamma_k$ без концевых точек) существует окрестность $B_\epsilon(z_0)$ такая, что $B_\epsilon(z_0)\setminus\Gamma_k \subset G$. Причем каждая точка $z_0 \in\buildrel\circ\over\Gamma_k$ называется внутренней точкой разреза.
\end{enumerate}
\MoreInsert{1}%FOR EXAMPLE
\lipsum[1-2]%FOR EXAMPLE
\fi

%%%%%%%%%%%%%%%%%%%%%%%%%%%%%%%%%%%%%%%%%%%%%%%%%%%%%%%%%%%%%%%%%%%
%%%%%%%%%%%%%%%%%%           PACKAGES            %%%%%%%%%%%%%%%%%% 
%%%%%%%%%%%%%%%%%%%%%%%%%%%%%%%%%%%%%%%%%%%%%%%%%%%%%%%%%%%%%%%%%%%

\usepackage{graphicx}
\usepackage[font=footnotesize]{caption}
\usepackage{microtype}
\input{insbox.tex}
\usepackage{threeparttable}
\usepackage{changepage}
\usepackage{xargs}
\usepackage{wrapfig}
\usepackage{printlen}   

\newcounter{ct}

\newlength\imageheight

\newcount\narrowlinect
\narrowlinect=0\relax 

%%%%%%%%%%%%%%%%%%%%%%%%%%%%%%%%%%%%%%%%%%%%%%%%%%%%%%%%%%%%%%%%%%%
%%%%%%%%%%%%%%%%%%           COMMANDS            %%%%%%%%%%%%%%%%%% 
%%%%%%%%%%%%%%%%%%%%%%%%%%%%%%%%%%%%%%%%%%%%%%%%%%%%%%%%%%%%%%%%%%%

\newcommand{\usepict}[3][0]{
\vspace*{-\baselineskip}
\begin{center}
\captionsetup{type=figure}\includegraphics[height=#3\textheight]{pictures/#2}\captionof{figure}{}\label{#2}
\end{center}
\vspace*{-#1\baselineskip}}

%%%%%%%%%%%%%%%%%%%%%%%%%%%%%%%%%%%%%%%%%%%%%%%%%%%%%%%%%%%%%%%%%%%

\newcommand\insertpicture[3][5]{%
\strictpagecheck%
\checkoddpage%
\ifoddpage%
\InsertBoxR{0}{\begin{threeparttable}\begin{tabular}{c@{}}\captionsetup{type=figure}\includegraphics[width=#3\textwidth]{pictures/#2}\end{tabular}\captionof{figure}{}\label{#2}\end{threeparttable}}[#1]%
\else%
\InsertBoxL{0}{\begin{threeparttable}\begin{tabular}{c@{}}\captionsetup{type=figure}\includegraphics[width=#3\textwidth]{pictures/#2}\end{tabular}\captionof{figure}{}\label{#2}\end{threeparttable}}[#1]%
\fi%
}

%%%%%%%%%%%%%%%%%%%%%%%%%%%%%%%%%%%%%%%%%%%%%%%%%%%%%%%%%%%%%%%%%%%

\newcommand\proofpicture[4][5]{%
\makebox[\textwidth-\widthof{\textit{Доказательство. }}][s]{#4}\par%
\insertpicture[#1]{#2}{#3}%
\noindent%
}

%%%%%%%%%%%%%%%%%%%%%%%%%%%%%%%%%%%%%%%%%%%%%%%%%%%%%%%%%%%%%%%%%%%

\newlength\wrapskiplength
\newlength\itemskiplenght
\newlength\picturewidth

\newlength\leftinsertbord
\newlength\rightinsertbord

\newcommand*\wrapitem{%
	\apptocmd\labelitemi{\hskip-\wrapskiplength}{}{}% add a correction
	\apptocmd\labelenumi{\hskip-\wrapskiplength}{}{}% add a correction
	\item
	\patchcmd\labelenumi{\hskip-\wrapskiplength}{}{}{}% remove the added hskip
	\patchcmd\labelitemi{\hskip-\wrapskiplength}{}{}{}% remove the added hskip
}

\newlength\InsertListPrevWidth
\makeatletter
\newcommand{\InsertListR}[3][0]
{%
	\mbox{}%
	\vspace*{-\baselineskip}%
	\setlength{\leftskip}{\leftmargin}%
	\InsertBoxR{#2}{\hskip-\leftmargin#3\hskip\leftmargin}[#1]%
	\global\InsertListPrevWidth\@framewidth
	
	\setlength{\leftinsertbord}{0cm}
	\setlength{\rightinsertbord}{\InsertListPrevWidth}
}%

\newcommand{\InsertListL}[3][0]{%
	\mbox{}%
	\vspace*{-\baselineskip}%
	\setlength{\leftskip}{0cm}%
	\InsertBoxL{#2}{\hskip0\leftmargin#3\hskip\leftmargin}[#1]%
	\global\InsertListPrevWidth\@framewidth
	
	\setlength{\leftinsertbord}{\picturewidth}
	\addtolength{\leftinsertbord}{\leftmargin}
	\addtolength{\leftinsertbord}{\labelsep}
	\setlength{\rightinsertbord}{0cm}
}%

\newcommand*\contitem[1][\the\count1]
{%
	\item
	\bgroup  % to keep some changes local
	% let's calculate parameters for \ParShape
	\def\reserved@a{#1}%
	\def\reserved@b{\the\count1}%
	\ifx\reserved@a\reserved@b
	% Below calculation is copied from \@@InsertBox of insbox
	\dimen0 = \@wherebottom
	\advance \dimen0 by -\pagetotal
	\divide \dimen0 by \baselineskip
	\count1 = \dimen0
	\advance \count1 by 1
	\advance \count1 by -\@numnormal
	\fi
	\MoreInsert{#1}%
	\egroup
}
\newcommand*\EndInsert
{%
	\@restore@
	\@ifstar{\@afterindentfalse\@afterheading}{}%
}
\newcommand*\MoreInsert[1]
{%
	\ParShape = 2
	{#1}   {\the\leftinsertbord}  {\the\rightinsertbord}
	1      0cm   0cm
}
\makeatother

\makeatletter
\newif\ifwide
\patchcmd\enitkv@enumitem@wide
{\enit@align@left}% <- Search
{\widetrue\enit@align@left}% <- Replace
{}{\FailedToPatch}
\makeatother

\newcommand{\itempicture}[3][5]{	
	\strictpagecheck%
	\checkoddpage%
	\ifoddpage%
	\setlength{\wrapskiplength}{-\leftmargin}
	\ifwide
	\setlength{\wrapskiplength}{-\labelsep}
	\else\fi
	\setlength{\picturewidth}{0cm}
	\wrapitem
	\InsertListR[#1]{0}{\begin{threeparttable}\begin{tabular}{c@{}}\captionsetup{type=figure}\includegraphics[width=#3\textwidth]{#2}\end{tabular}\captionof{figure}{}\label{#2}\end{threeparttable}}
	\ifwide
	\setlength{\itemskiplenght}{\widthof{\hbox{\number\value{enumi}.\,}}}
	\hskip\itemskiplenght
	\else\fi
	\else%
	\setlength{\picturewidth}{\widthof{\hbox{\begin{threeparttable}\begin{tabular}{c@{}}\captionsetup{type=figure}\includegraphics[width=#3\textwidth]{#2}\end{tabular}\captionof{figure}{}\label{#2}\end{threeparttable}}}}
	\ifwide
	\item[]
	\else
	\setlength{\wrapskiplength}{\picturewidth}
	\addtolength{\wrapskiplength}{\labelsep}
	\wrapitem
	\fi
	\InsertListL[#1]{0}{\begin{threeparttable}\begin{tabular}{c@{}}\captionsetup{type=figure}\includegraphics[width=#3\textwidth]{#2}\end{tabular}\captionof{figure}{}\label{#2}\end{threeparttable}}
	\ifwide
	\addtocounter{enumi}{1}\hskip-0.5px\number\value{enumi}.~\hskip1px
	\else\fi
	\fi	
}





























































\iffalse
%%%%%%%%%%%%%%%%%%%%%%%%%%%%%%%%%%%%%%%%%%%%%%%%%%%%%%%%%%%%%%%%%%%
%%%%%%%%%%%%%%%%%%%%%%%%%%%%%%%%%%%%%%%%%%%%%%%%%%%%%%%%%%%%%%%%%%%
%%%%%%%%%%%%%%%%%%%%%%%%%%%%%%%%%%%%%%%%%%%%%%%%%%%%%%%%%%%%%%%%%%%

%%%        OLD AND OUTDATED CONSTRACTIONS ARE THERE             %%%

%%%%%%%%%%%%%%%%%%%%%%%%%%%%%%%%%%%%%%%%%%%%%%%%%%%%%%%%%%%%%%%%%%%
%%%%%%%%%%%%%%%%%%%%%%%%%%%%%%%%%%%%%%%%%%%%%%%%%%%%%%%%%%%%%%%%%%%
%%%%%%%%%%%%%%%%%%%%%%%%%%%%%%%%%%%%%%%%%%%%%%%%%%%%%%%%%%%%%%%%%%%


% \wrappicture : To use it when you want to wrap a picture inside list environemnts:
% \wrappicture[number of narrow lines of item]{textoffirstitem}[number of narrow lines in the second item]{ratio of textwidth}[text in the second item]{picture}[number of untouched lines]
% Example from chapter35, definition14:
\iffalse
\begin{enumerate}
	\wrappicture{
		\item Кривая $\Gamma_k$ "--- \textit{правильной гладкой компонентой} $\Gamma$, если в каждой окрестности каждой точки $z_0 \in \Gamma_k$ находятся как точки из области $G$, так и из $\bbC \setminus G\cup\Gamma$.}
	{0.3}[
	\item Кривая $\Gamma_k$ "--- \textit{разрезом}, если для каждой точки $z_0 \in\buildrel\circ\over\Gamma_k$ (обозначение $\buildrel\circ\over\Gamma_k$ "--- кривая $\Gamma_k$ без концевых точек) существует окрестность $B_\epsilon(z_0)$ такая, что $B_\epsilon(z_0)\setminus\Gamma_k \subset G$. Причем каждая точка $z_0 \in\buildrel\circ\over\Gamma_k$ называется внутренней точкой разреза.]
	{ch35pict3}
\end{enumerate}
\fi

\newcommandx{\wrappicture}[7][1=12,3=8,5=\mbox{},7=0]
{\vspace*{-\baselineskip}
\parbox[t]{\dimexpr\textwidth-\leftmargin}
{%
\vspace{-2.5mm}
\begin{wrapfigure}[#1]{o}{#4\textwidth}
\centering
\vspace{-\baselineskip}
\InsertBoxL{#7}
{\begin{threeparttable}%
\begin{tabular}{c@{}}\captionsetup{type=figure}\includegraphics[width=#4\textwidth]{pictures/#6}\end{tabular}\captionof{figure}{}\label{#6}\end{threeparttable}}%[#extraargument?]%  
\end{wrapfigure}#2
\parbox[t]{\dimexpr\textwidth-\leftmargin}{%
\vspace{-2.5mm}
\begin{wrapfigure}[#3]{r}{#4\textwidth}
\end{wrapfigure}
#5}}}
		
%%%%%%%%%%%%%%%%%%%%%%%%%%%%%%%%%%%%%%%%%%%%%%%%%%%%%%%%%%%%%%%%%%%		



\newcommand{\addpictureitemL}[4][9]{
	\refstepcounter{pictnumber}
	\parbox[t]{\dimexpr\textwidth-\leftmargin}{%
		\vspace{-2.5mm}
		\begin{wrapfigure}[#1]{l}{#3\textwidth}
			\centering
			\vspace{-\baselineskip}
			\includegraphics[width=#3\textwidth]{#2}
			\caption*{\footnotesize Рис. \thepictnumber}
		\end{wrapfigure}#4}}

\newcommand{\addpictureitemR}[4][9]{
	\refstepcounter{pictnumber}
	\parbox[t]{\dimexpr\textwidth-\leftmargin}{%
		\vspace{-2.5mm}
		\begin{wrapfigure}[#1]{r}{#3\textwidth}
			\centering
			\vspace{-\baselineskip}
			\includegraphics[width=#3\textwidth]{#2}
			\caption*{\footnotesize Рис. \thepictnumber}
		\end{wrapfigure}#4}}

\newcommand{\addpictureitem}[4][9]{%
	\strictpagecheck%
	\checkoddpage%
	\ifoddpage%
	\addpictureitemR[#1]{#2}{#3}{#4}%
	\else%
	\addpictureitemL[#1]{#2}{#3}{#4}%
	\fi%
}      



%%%%%%%%%%%%%%%%%%%%%%%%%%%%%%%%%%%%%%%%%%%%%%%%%%%%%%%%%%%%%%%%%%%
%%%        OLD AND OUTDATED CONSTRACTIONS ARE THERE             %%%
%%%%%%%%%%%%%%%%%%%%%%%%%%%%%%%%%%%%%%%%%%%%%%%%%%%%%%%%%%%%%%%%%%%

%RULES. Pls check it everytime.
%NOT WRAPPED BUT VERY EASY.
%\usepict[〈height of the textheight〉]{pictures/〈picture〉}	

%for ez text
%\mywrappict[〈number of narrow lines〉] [〈overhang〉] {〈width〉}

% IT WORKS EVERYWHERE EXCEPT LISTS and beginning of theorems!
%\addpicture[〈number of narrow lines〉]{〈exactly picture〉}[〈after line〉]{〈width of \textwidth picture will occupy〉}

%It is more complex and more cool variant than previous one.
%\InsertPict[〈number of narrow lines〉]{〈exactly picture〉}[〈after line〉]{〈width of \textwidth picture will occupy〉}
% do not forget about super function for lists: \wrapitem - just put it 

%For Theorems and smth like proof. ONE PICTURE TO ONE text box
%\addmppict[〈this partial of 3 argument picture will occupy, correction!〉]{〈exactly picture〉}{〈width of \textwidth picture will occupy〉]}〈text of theorem〉}

%SEE THE MOST POWERFUL CONSTRUCTION, but it is hard to set.
%\wrappicture[number of narrow lines of item]{textoffirstitem}[number of narrow lines in the second item]{ratio of textwidth}[text in the second item]{picture}[number of untouched lines]

% NOW IT IS USELESS, but can work if nothing helps.
%\addpictureitem[number of narrow lines]{exactly picture}{width of \textwidth will picture occupy}{text}

%%%%%%%%%%%%%%%%%%%%%%%%%%%%%%%%%%%%%%%%%%%%%%%%%%%%%%%%%%%%%%%%%%%
%%%%%%%%%%%%%%%%%%%%%%%%%%%%%%%%%%%%%%%%%%%%%%%%%%%%%%%%%%%%%%%%%%%
%%%%%%%%%%%%%%%%%%%%%%%%%%%%%%%%%%%%%%%%%%%%%%%%%%%%%%%%%%%%%%%%%%%
%%%%%%%%%%%%%%%%%%%%%%%%%%%%%%%%%%%%%%%%%%%%%%%%%%%%%%%%%%%%%%%%%%%
%%%%%%%%%%%%%%%%%%%%%%%%%%%%%%%%%%%%%%%%%%%%%%%%%%%%%%%%%%%%%%%%%%%
  
\newcommand{\addmppictL}[4][1]{%
\noindent
\begin{minipage}[t]{#3\textwidth}
\addpicture{#2}{#1}
\end{minipage}%
\hfill%
\begin{minipage}[t]{\dimexpr(1\textwidth-#3\textwidth-2mm)}
#4
\end{minipage}}

\newcommand{\addmppictR}[4][1]{%
\noindent
\begin{minipage}[t]{\dimexpr(1\textwidth-#3\textwidth-2mm)}
#4
\end{minipage}%
\hfill%
\begin{minipage}[t]{#3\textwidth}
\addpicture{#2}{#1}
\end{minipage}}

\newcommand{\addmppict}[4][1]{%
\strictpagecheck%
\checkoddpage%
\ifoddpage
\addmppictR[#1]{#2}{#3}{#4}
\else%
\addmppictL[#1]{#2}{#3}{#4}
\fi%
}

  
%%%%%%%%%%%%%%%%%%%%%%%%%%%%%%%%%%%%%%%%%%%%%%%%%%%%%%%%%%
%%%%%%%%%%%%%%%%%%%%%%%%%%%%%%%%%%%%%%%%%%%%%%%%%%%%%%%%%%  
  
\newcommandx\addpictureR[4][1=0,3=0]{\refstepcounter{pictnumber}%
\settoheight\imageheight{\includegraphics[width=#4\textwidth]{#2}}
\narrowlinect=\imageheight\relax
\setcounter{ct}{\numexpr((\narrowlinect)/\baselineskip+2)\relax}
\InsertBoxR{#3}{\begin{threeparttable}%
\begin{tabular}{c@{}}\includegraphics[width=#4\textwidth]{#2}\end{tabular}%
\caption*{\footnotesize Рис. \thepictnumber}\end{threeparttable}}[\numexpr(\value{ct}/2+#1+1)]}

\newcommandx\addpictureL[4][1=0,3=0]{\refstepcounter{pictnumber}%
\settoheight\imageheight{\includegraphics[width=#4\textwidth]{#2}}
\narrowlinect=\imageheight\relax
\setcounter{ct}{\numexpr((\narrowlinect)/\baselineskip+2)\relax}
\InsertBoxL{#3}{\begin{threeparttable}%
\begin{tabular}{c@{}}\includegraphics[width=#4\textwidth]{#2}\end{tabular}%
\caption*{\footnotesize Рис. \thepictnumber}\end{threeparttable}}[\numexpr(\value{ct}/2+#1+1)]}

\newcommandx{\addpicture}[4][1=0,3=0]{%
\strictpagecheck%
\checkoddpage%
\ifoddpage
\addpictureR[#1]{#2}[#3]{#4}
\else%
\addpictureL[#1]{#2}[#3]{#4}
\fi%
}

%%%%%%%%%%%%%%%%%%%%%%%%%%%%%%%%%%%%%%%%%%%%%%%%%%%%%%%%%%
%%%%%%%%%%%%%%%%%%%%%%%%%%%%%%%%%%%%%%%%%%%%%%%%%%%%%%%%%%



%%%%%%%%%%%%%%%%%%%%%%%%%%%%%%%%%%%%%%%%%%%%%%%%%%%%%%%%%%
%%%%%%%%%%%%%%%%%%%%%%%%%%%%%%%%%%%%%%%%%%%%%%%%%%%%%%%%%%
          
%you can add CT-FUNCTION HERE  IF IT IS NECESSARY
          

%%%%%%%%%%%%%%%%%%%%%%%%%%%%%%%%%%%%%%%%%%%%%%%%%%%%%%%%%%%
%%%%%%%%%%%%%%%%%%%%%%%%%%%%%%%%%%%%%%%%%%%%%%%%%%%%%%%%%%%
       
    \usepackage{booktabs}
    %\usepackage{nccmath}
    %\usepackage{etoolbox}
      
    %\newcommand*{\wrapitem}{\apptocmd{\labelenumi}{\hskip\leftmargin}{}{}\item\apptocmd{\labelenumi}{\hski5p-\leftmargin}{}{}}
    %
    \newcommandx{\InsertPictL}[4][1=0,3=0]{\refstepcounter{pictnumber}%
    \setlength{\leftskip}{\leftmargin}\mbox{}\vspace*{-\baselineskip}%
    \InsertBoxL{#1}{\begin{threeparttable}%
\begin{tabular}{c@{}}\includegraphics[width=#4\textwidth]{#2}\end{tabular}%
\caption*{{\footnotesize Рис. }\thepictnumber}\end{threeparttable}}[#3]\par \hspace{\itemindent}
    }%
    
    \newcommandx{\InsertPictR}[4][1=0,3=0]{\refstepcounter{pictnumber}%
    \mbox{}\vspace*{-\baselineskip}\setlength{\leftskip}{\leftmargin}%
    \InsertBoxR{#1}{\hskip-\leftmargin\begin{threeparttable}%
\begin{tabular}{c@{}}\includegraphics[width=#4\textwidth]{#2}\end{tabular}%
\caption*{\footnotesize Рис. \thepictnumber}\end{threeparttable}\hskip\leftmargin}[#3]
    }%

\newcommandx{\InsertPict}[4][1=0,3=0]{%
\strictpagecheck%
\checkoddpage%
\ifoddpage
\InsertPictR[#1]{#2}[#3]{#4}
\else%
\InsertPictL[#1]{#2}[#3]{#4}
\fi%
}    


\fi
