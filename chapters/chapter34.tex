\chapter{Центральная предельная теорема для независимых одинаково распределенных случайных величин с конечной дисперсией.}

\begin{thm}
Если случайные величины $\xi_1$,~$\xi_2$, \dots,~$\xi_n$, \dots независимы, одинаково распределены и имеют конечную дисперсию, то при $n\to +\infty$ равномерно по $x\in(-\infty;+\infty)$
$$
\bbP \left(\frac{\xi_1+\xi_2+\dots+\xi_n -na}{\sigma \sqrt{n}} < x\right) \to \frac{1}{\sqrt{2\pi}}\int_{-\infty}^{x} e^{-u^2/2}\,du,
$$
где $a = \bbE \xi_k$, $\sigma = \bbD\xi_k$, $k\in\overline{1,n}$
\end{thm}

 Прошу прощения, доказательство и подводящие моменты пока что не написаны. В текущий момент вы можете посмотреть \S 2 Главы 8 в книге Чистякова. Если кто-то хочет проявить энтузиазм и написать эту главу в \LaTeX "--- буду рад помощи. 

%%%%%%%%%%%%%%%%%%%%%%%%%%%%%%%%%%%%
%%%%%%%%%%%%%%%%%%%%%%%%%%%%%%%%%%%%
%%%%%%%%%%%%%%%%%%%%%%%%%%%%%%%%%%%%

\iffalse
\section{Схема Бернулли}

\textit{Схема независимых испытаний, в которой каждое испытание может закончиться только одним из двух исходов, называется схемой Бернулли.}

Обычно эти исходы называют <<успехом>> и <<неудачей>>, а их вероятности обозначают $p$ и $q = 1 - p$ $(0 \le p \le 1)$ соответственно.

Наступление или ненаступление события~$A$ в испытаниях с разными номерами для схемы Бернулли независимы. Значит, в силу теоремы умножения вероятностей, вероятность того, что событие $A$ наступит в $m$ определенных испытаниях (например, в испытаниях с номерами $s_1, s_2, \ldots, s_m$), а при остальных $n - m$ не наступит, равна $p^mq^{n - m}$. Эта вероятность не зависит от  расположения номеров $s_1, s_2, \ldots, s_m$.

Простейшая задача, относящаяся к схеме Бернулли, состоит в определении вероятности $\bbP_n(m)$ того, что в $n$ испытаниях событие $A$ произойдет $m$ раз ($0 \le m \le n$).

Мы только что нашли, что вероятность того, что событие $A$ наступит в испытаниях с определенными $m$ номерами, а в остальных не наступит равна $p^mq^{n - m}$. По теореме сложения искомая вероятность равна сумме только что вычисленных вероятностей для всех различных способов размещения $m$ появлений события $A$ и $n - m$ непоявлений среди $n$ испытаний. Число таких способов известно из комбинаторики. Это число обозначается  $C_n^m$ или $\binom n m $. И оно равно	
$$
\binom n m = \frac{n!}{m!\;(n - m)!}
$$
и, следовательно,
\begin{equation} \label{ch32.1.1eq1}
\bbP_n(m) = \binom n m p^mq^{n - m} \quad (m = 0,1,2,\ldots,n).
\end{equation}

\begin{thm} [Бернулли\rindex{теорема!Бернулли}]
Пусть $\mu_n$ --- число успехов в $n$ испытаниях Бернулли и $p$ "--- вероятность успеха в каждом отдельном испытании. Тогда $\forall \epsilon > 0$
\begin{equation} \label{ch32.2eq7}
\lim_{n \to \infty} \bbP \left( \left| \frac{\mu_n}{n} - p \right| < \epsilon \right) = 1.
\end{equation}
\end{thm}

\begin{proof}
Для доказательства этой теоремы воспользуемся представлением $\mu_n$ в виде суммы $n$ индикаторов: $\mu_n = \xi_1 + \xi_2 + \ldots + \xi_n$, где $\xi_k = 1$, если в $k$-м испытании был успех, и $\xi_k  = 0$ в противном случае.

Так как $\xi_k$, $k = \overline{1,n}$, независимы, одинаково распределены $(\bbP( \xi_k = 1) = p, \,\bbP( \xi_k = 0) = 1 - p = q)$, то дисперсии случайных величин $\xi_k$ существуют и $\bbE  \xi_k = p$, то теорема Бернулли сразу следует из теоремы $\ref{ch32.2T4}$.
\end{proof}

\section{Предельная теорема Пуассона}
\begin{thm} [Пуассона\rindex{теорема!Пуассона}]
Если $n \to \infty$ и $p \to 0$ так, что $np \to \lambda$, $0 < \lambda < \infty$, то
$$
\bbP(\mu_n = m) = \bbP_n(m) = \binom n m p^mq^{n - m} \to p_m(\lambda) = \frac{\lambda^m}{m!} e^{-\lambda}
$$
при любом постоянном $m$,  $m = \overline{0,\dots}$.
\end{thm}

\begin{proof}
Положив $np = \lambda_n$, представим вероятность $\bbP_n(m)$ в виде
\begin{multline*}
\bbP(\mu_n = m) = \frac{n(n - 1)\ldots(n - m + 1)}{m!} \left( \frac{\lambda_n}{n}\right)^m\left( 1 - \frac{\lambda_n}{n}\right)^{n - m} = \\
= \frac{\lambda_n^m}{m!} \left( 1 - \frac{\lambda_n}{n}\right)^n \left( 1 - \frac{1}{n}\right)\left(1 - \frac{2}{n} \right) \ldots \left( 1 - \frac{m - 1}{n}\right)\left(1 - \frac{\lambda_n}{n} \right)^{-m}.
\end{multline*}

Отсюда при $n \to \infty$ получим утверждение теоремы.
\end{proof}

Таким образом, при больших $n$ и малых $p$ мы можем воспользоваться приближенной формулой
$$
\bbP(\mu_n = m) \approx \frac{(\lambda_n)^m}{m!} e^{-\lambda_n},	\lambda_n = np.
$$
\fi