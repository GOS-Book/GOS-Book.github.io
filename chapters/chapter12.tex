\chapter{Равномерная сходимость функциональных последовательностей и рядов. Непрерывность, интегрируемость и дифференцируемость суммы равномерно сходящегося ряда.}

\section{Равномерная сходимость функциональных последовательностей и рядов}

\subsection{Сходимость функциональных последовательностей и рядов}

Пусть задана последовательность функций $\{f_n\}$
\begin{equation} \label{ya9ch1_1n1}
f_n(x),\ n\in \bbN, 
\end{equation}
определенных на некотором множестве $E$.
\begin{defn} Функциональная последовательность называется \textit{сходящейся в точке} $x_0 \in E$, если числовая последовательность $\{f_n(x_0)\}$ сходится.

Последовательность $\{f_n\}$ называется сходящейся на множестве $E$, если она сходится в каждой точке множества $E$.

Если
\begin{equation} \label{ya9ch1_1n2}
\lim_{n \to \infty} f_n(x) = f(x)\quad \forall x\in E,
\end{equation}
то говорят, что \textit{последовательность \eqref{ya9ch1_1n1} на множестве $E$ сходится к функции} $f(x)$, и пишут <<$f_n(x)\to f(x)$ на $E$>> или <<$f_n(x) \xrightarrow{E} f(x)$ при $n \to +\infty $>>.

Эта функция $f(x)$ называется пределом или предельной функцией последовательности. В этом случае иногда говорят, что \textit{последовательность $\{f_n\}$ сходится к функции~$f(x)$ поточечно}.
\end{defn}

Соответствующим образом определяется и сходимость функционального ряда
\begin{equation} \label{ya9ch1_1n3}
\sum_{n=1}^{\infty}\limits u_n(x)
\end{equation}
члены которого определены на множестве $E$.
\begin{defn} Ряд \eqref{ya9ch1_1n3} называется \textit{сходящимся в точке} $x_0 \in E$, если числовой ряд $\sum_{n=1}^{\infty}\limits u_n(x_0)$ сходится.

Ряд \eqref{ya9ch1_1n3} \textit{называется сходящимся на множестве} $E$, если он сходится в любой точке множества $E$. Если
\begin{equation} \label{ya9ch1_1n4}
f(x)=\sum_{n=1}^{\infty}\limits u_n(x) \quad\forall x \in E,
\end{equation}
то функция $f(x)$ называется \textit{суммой ряда} \eqref{ya9ch1_1n3}.
\end{defn}

\section{Непрерывность, интегрируемость и дифференцируемость суммы равномерно сходящегося ряда}

\subsection{Непрерывность суммы равномерно сходящегося ряда}

\begin{thm}
	\label{ch12:th:continuity}
	Если все члены ряда~\eqref{ya9ch1_1n3} --- непрерывные на отрезке $[a, b]$ функции,
	а ряд~\eqref{ya9ch1_1n3} сходится равномерно на $[a, b]$, то его сумма $S(x)$
	также непрерывна на отрезке $[a, b]$.
\end{thm}
\begin{proof}
	Пусть $x_0$ --- произвольная точка отрезка $[a, b]$. Для определенности будем считать,
	что $x_0 \in (a, b)$.

	Нужно доказать, что функция
	$$
		S(x) = \sum_{n=1}^{\infty} u_n(x)
	$$
	непрерывна в точке $x_0$, т. е.
	\begin{equation}
		\label{ch12:eq:1}
		\forall \epsilon > 0 \quad \exists \delta = \delta(\epsilon) > 0 \cquad
			\forall x \in U_{\delta}(x_0) \quad |S(x) - S(x_0)| < \epsilon,
	\end{equation}
	где $U_{\delta}(x_0) = (x_0 - \delta, x_0 + \delta) \subset [a, b]$.

	По условию $S_n(x) \rightrightarrows S(x)$, $x \in [a, b]$,
	где $S_n(x) = \sum_{k=1}^{n}\limits u_k(x)$, т. е.
	\begin{equation}
		\label{ch12:eq:2}
		\forall \epsilon > 0 \quad \exists N_{\epsilon} \cquad \forall n \ge N_{\epsilon} \quad
			\forall x \in [a, b] \quad |S(x) - S_n(x)| < \frac{\epsilon}{3}
	\end{equation}
	Фиксируем номер $n_0 \ge N_{\epsilon}$. Тогда из~\eqref{ch12:eq:2} при $n = n_0$ получаем
	\begin{equation}
		\label{ch12:eq:3}
		|S(x) - S_{n_0}(x)| < \frac{\epsilon}{3}
	\end{equation}
	и, в частности, при $x = x_0$ находим
	\begin{equation}
		\label{ch12:eq:4}
		|S(x_0) - S_{n_0}(x_0)| < \frac{\epsilon}{3}
	\end{equation}
	Функция $S_{n_0}(x)$ непрерывна в точке $x_0$ как сумма конечного числа непрерывных функций
	$u_k(x)$, $k = \overline{1, n_0}$. По определению непрерывности
	\begin{equation}
		\label{ch12:eq:5}
		\forall \epsilon > 0 \quad \exists \delta = \delta(\epsilon) > 0 \cquad
			\forall x \in U_\delta(x_0) \subset [a, b] \quad |S_{n_0}(x) - S_{n_0}(x_0)| < \frac{\epsilon}{3}
	\end{equation}
	Воспользуемся равенством
	$$
		S(x) - S(x_0) = (S(x) - S_{n_0}(x)) + (S_{n_0}(x) - S_{n_0}(x_0)) + (S_{n_0}(x_0) - S(x_0)).
	$$
	Из этого равенства, используя оценки~\eqref{ch12:eq:3}--\eqref{ch12:eq:5}, получаем
	$$
		|S(x) - S(x_0)| = |S(x) - S_{n_0}(x)| + |S_{n_0}(x) - S_{n_0}(x_0)| + |S_{n_0}(x_0) - S(x_0)| < \epsilon
	$$
	для любого $x \in U_\delta(x_0) \subset [a, b]$, т. е. справедливо утверждение~\eqref{ch12:eq:1}.

	Так как $x_0$ --- произвольная точка отрезка $[a, b]$, то функция $S(x)$ непрерывна на отрезке $[a, b]$.
\end{proof}

\begin{cons}
	Согласно теореме~\eqref{ch12:th:continuity}
	$$
		\lim_{x \to x_0} \sum_{n = 1}^{\infty} u_n(x) = \sum_{n = 1}^{\infty} \lim_{x \to x_0} u_n(x),
	$$
	т. е. при условиях теоремы~\eqref{ch12:th:continuity} возможен почленный предельный переход.
\end{cons}

\subsection{Почленное интегрирование функционального ряда}

\begin{thm}
	\label{ch12:th:integration}
	Если все члены ряда~\eqref{ya9ch1_1n3} --- непрерывные на отрезке $[a, b]$ функции,
	а ряд~\eqref{ya9ch1_1n3} сходится равномерно на $[a, b]$, то ряд
	\begin{equation}
		\label{ch12:eq:6}
		\sum_{n=1}^{\infty} \int_a^x u_n(t)dt
	\end{equation}
	также равномерно сходится на $[a, b]$, и если
	\begin{equation}
		\label{ch12:eq:7}
		S(x) = \sum_{n=1}^{\infty} u_n(x),
	\end{equation}
	то
	\begin{equation}
		\label{ch12:eq:8}
		\int_a^x S(t)dt = \sum_{n=1}^{\infty} \int_a^x u_n(t)dt, \quad x \in [a, b],
	\end{equation}
	т. е. ряд~\eqref{ch12:eq:7} можно почленно интегрировать
\end{thm}
\begin{proof}
	По условию ряд~\eqref{ch12:eq:7} сходится равномерно к $S(x)$ на отрезке $[a, b]$,
	т. е. $S_n(x) = \sum_{k=1}^{n}\limits u_k(x) \rightrightarrows S(x), x \in [a, b]$.
	Это означает, что
	\begin{equation}
		\label{ch12:eq:9}
		\forall \epsilon > 0 \quad \exists N_{\epsilon} \cquad \forall n \ge N_{\epsilon} \quad
			\forall t \in [a, b] \quad |S(t) - S_n(t)| < \frac{\epsilon}{b - a}
	\end{equation}
	Пусть $\sigma(x) = \int_a^x\limits S(t)dt$, а $\sigma_n(x) = \sum_{k=1}^n\limits \int_a^x\limits u_k(t)dt$ ---
	$n$-я частичная сумма ряда~\eqref{ch12:eq:6}.

	Функции $u_k(x), k \in \bbN$, по условию непрерывны на отрезке $[a, b]$, и поэтому
	они интегрируемы на $[a, b]$. Функция $S(x)$ также интегрируема на $[a, b]$, так как
	она непрерывна на этом отрезке (теорема~\ref{ch12:th:continuity}).
	Используя свойства интеграла, получаем
	$$
		\sigma_n(x) = \int_a^x \sum_{k=1}^n u_k(t)dt = \int_a^x S_n(t)dt.
	$$
	Следовательно,
	$$
		\sigma(x) - \sigma_n(x) = \int_a^x (S(t) - S_n(t))dt,
	$$
	откуда в силу условия~\eqref{ch12:eq:9} получаем
	$$
		|\sigma(x) - \sigma_n(x)| < \frac{\epsilon}{b - a} \int_a^x dt = \frac{\epsilon}{b - a} (x - a) \le \epsilon,
	$$
	причем это неравенство выполняется для всех $n \ge N_{\epsilon}$ и для всех $x \in [a, b]$.
	Это означает, что ряд~\eqref{ch12:eq:6} сходится равномерно на отрезке $[a, b]$,
	и выполняется равенство~\eqref{ch12:eq:8}.
\end{proof}

\subsection{Почленное дифференцирование функционального ряда}

\begin{thm}
	Если функции $u_n(x), n \in \bbN$, имеют непрерывные производные на отрезке $[a, b]$, ряд
	\begin{equation}
		\label{ch12:eq:10}
		\sum_{n=1}^{\infty} u'_n(x)
	\end{equation}
	сходится равномерно на отрезке $[a, b]$, а ряд
	\begin{equation}
		\label{ch12:eq:11}
		\sum_{n=1}^{\infty} u_n(x)
	\end{equation}
	сходится хотя бы в одной точке $x_0 \in [a, b]$, т. е. сходится ряд
	\begin{equation}
		\label{ch12:eq:12}
		\sum_{n=1}^{\infty} u_n(x_0),
	\end{equation}
	то ряд~\eqref{ch12:eq:11} сходится равномерно на отрезке $[a, b]$,
	и его можно почленно дифференцировать, т. е.
	\begin{equation}
		\label{ch12:eq:13}
		S'(x) = \sum_{n=1}^{\infty} u'_n(x),
	\end{equation}
	где
	\begin{equation}
		\label{ch12:eq:14}
		S(x) = \sum_{n=1}^{\infty} u_n(x)
	\end{equation}
\end{thm}
\begin{proof}
	Обозначим через $\tau(x)$ сумму ряда~\eqref{ch12:eq:10}, т. е.
	\begin{equation}
		\label{ch12:eq:15}
		\tau(x) = \sum_{n=1}^{\infty} u'_n(x)
	\end{equation}
	По теореме~\ref{ch12:th:integration} ряд~\eqref{ch12:eq:15} можно почленно интегрировать, т. е.
	\begin{equation}
		\label{ch12:eq:16}
		\int_{x_0}^x \tau(t)dt = \sum_{n=1}^{\infty} \int_{x_0}^x u'_n(t)dt,
	\end{equation}
	где $x_0, x \in [a, b]$, причем ряд~\eqref{ch12:eq:16} сходится равномерно на отрезке $[a, b]$.
	Так как $\int_{x_0}^x\limits u'_n(t)dt = u_n(x) - u_n(x_0)$, то равенство~\eqref{ch12:eq:16}
	можно записать в виде
	\begin{equation}
		\label{ch12:eq:17}
		\int_{x_0}^x \tau(t)dt = \sum_{n=1}^{\infty} \upsilon_n(x),
	\end{equation}
	где
	\begin{equation}
		\label{ch12:eq:18}
		\upsilon_n(x) = u_n(x) - u_n(x_0).
	\end{equation}
	Ряд~\eqref{ch12:eq:17} сходится равномерно, а ряд~\eqref{ch12:eq:12} сходится
	(а значит, и равномерно сходится на $[a, b]$). Поэтому ряд~\eqref{ch12:eq:11}
	сходится равномерно на $[a, b]$ как разность равномерно сходящихся рядов.

	Из равенств \eqref{ch12:eq:17}, \eqref{ch12:eq:18} и \eqref{ch12:eq:14} следует, что
	\begin{equation}
		\label{ch12:eq:19}
		\int_{x_0}^x \tau(t)dt = S(x) - S(x_0)
	\end{equation}
	Так как функция $\tau(t)$ непрерывна на отрезке $[a, b]$ по теореме~\ref{ch12:th:continuity},
	то в силу свойств интеграла с переменным верхним пределом левая часть равенства~\eqref{ch12:eq:19}
	имеет производную, которая равна $\tau(x)$. Следовательно, правая часть~\eqref{ch12:eq:19} ---
	дифференцируемая функция, а её производная равна $S'(x)$. Итак, доказано, что $\tau(x) = S'(x)$,
	т. е. справедливо равенство~\eqref{ch12:eq:13} для всех $x \in [a, b]$
\end{proof}