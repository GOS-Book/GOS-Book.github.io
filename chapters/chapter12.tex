\chapter{Равномерная сходимость функциональных последовательностей и рядов. Непрерывность, интегрируемость и дифференцируемость суммы равномерно сходящегося ряда.}

\section{Равномерная сходимость функциональных последовательностей и рядов}

\subsection{Сходимость функциональных последовательностей и рядов}

Пусть задана последовательность функций $\{f_n\}$
\begin{equation} \label{ya9ch1_1n1}
f_n(x), n\in \bbN, 
\end{equation}
определенных на некотором множестве $E$.
\begin{defn} Функциональная последовательность называется \textit{сходящейся к точке} $x_0 \in E$, если числовая последовательность $\{f_n(x_0)\}$ сходится.

Последовательность $\{f_n\}$ называется сходящейся на множестве $E$, если она сходится в каждой точке множества $E$.

Если
\begin{equation} \label{ya9ch1_1n2}
\lim_{n \to \infty} f_n(x) = f(x) \fa x\in E,
\end{equation}
то говорят, что \textit{последовательность \eqref{ya9ch1_1n1} на множестве $E$ сходится к функции} $f(x)$, и пишут <<$f_n(x)\to f(x)$ на $E$>> или <<$f_n(x) \xrightarrow{E} f(x)$ при $n \to +\infty $>>.

Эта функция $f(x)$ называется пределом или предельной функцией последовательности. В этом случае иногда говорят, что \textit{последовательность $\{f_n\}$ сходится к функции $f(x)$ поточечно}.
\end{defn}

Соответствующим образом определяется и сходимость функционального ряда
\begin{equation} \label{ya9ch1_1n3}
\sum_{n=1}^{\infty}\limits u_n(x)
\end{equation}
члены которого определены на множестве $E$.
\begin{defn} Ряд \eqref{ya9ch1_1n3} называется \textit{сходящимся в точке} $x_0 \in E$, если числовой ряд $\sum_{n=1}^{\infty}\limits u_n(x_0)$ сходится.

Ряд \eqref{ya9ch1_1n3} \textit{называется сходящимся на множестве} $E$, если он сходится в любой точке множества $E$. Если
\begin{equation} \label{ya9ch1_1n4}
f(x)=\sum_{n=1}^{\infty}\limits u_n(x) \fa x \in E,
\end{equation}
то функция $f(x)$ называется \textit{суммой ряда} \eqref{ya9ch1_1n3}.
\end{defn}

\section{Непрерывность, интегрируемость и дифференцируемость суммы равномерно сходящегося ряда}