\chapter{Математическое ожидание и дисперсия случайной величины, их свойства.}
\section{Математическое ожидание и дисперсия случайной величины, их свойства}

\subsection{Случайные величины}

Пусть дано конечное вероятностное пространство $(\Omega,\mathcal A, P)$, где $P$ --- вероятности каждого из элементарных исходов из $\Omega$. Тогда \textit{случайной величиной} принято называть любую функцию $\xi\colon \Omega \to \bbR$.(т.е. случайному элементарному исходу ставится в соответствие совершенно конкретное значение.)

Абсолютно так же определяется случайная величина для бесконечного счетного вероятностого пространства, где $\Omega = {w_1,\dots, w_n,...}$.

Пусть теперь дано произвольное вероятностное пространство $(\Omega,\mathcal A, P)$, где $P$ --- вероятности каждого из элементарных исходов из $\Omega$. Тогда в этом общем случае:
\begin{defn}
\textit{Случайной величиной} называется действительная функция от элементарного события $\xi = \xi(w), \; w\in\Omega$, для которой при любом действительном $x$ множество $\{w: \xi(w) \le x\}$ принадлежит $\mathcal A$ (т. е. является событием) и для него определена вероятность $P(w: \xi(w) \le x)$, записываемая кратко $F_\xi(x)=P(\xi \le x)$. Эта вероятность, рассматриваемая как функция $x$, называется \textit{функцией распределения случайной величины $\xi$}.

Отметим ее свойства:
\begin{enumerate}
\item 
$0 \le F(x) \le 1 \fa x;$ 
\item
$F(x_1) \ le F(x_2), \text{если}\; x_1\le x_2;$
\item
$F(-\infty)=0, \; F(+\infty)=1;$
\item
$F(x+0)=F(x)$ -- непрерывна слева.
\end{enumerate}
Важным классом распределении вероятностей являются \textit{абсолютно непрерывные распределения}, задаваемые плотностью вероятности $p_\xi(x) = р(х)$, т. о. такой неотрицательной функцией $р(х)$, что для любого события $B\in\Omega$.
$$
P(\xi \in B)=\int_B p(x)\,dx,
$$
Тогда функция распределения $F_\xi(x)=\int_{-\infty}^{x}p(x)\,dx$, где $p(x)$ называют \textit{плотностью вероятности}, обладающая следующими свойствами:
\begin{enumerate}
\item 
$p(x)\le 0$
\item 
$\forall x_1,x_2: P(x_1\le\xi<x_2)=\int_{x_1}^{x_2}p(x)\,dx$
\item
$\int_{-\infty}^{+\infty}p(x)\,dx=1$
\end{enumerate}

Другой класс составляют \textit{дискретные распределения}, задаваемые конечным или счетным набором вероятностей $Р(\xi=x_k)$ для которых
$$
\sum\limit_k P(\xi=x_k)=1,
$$
тогда функция распределения $F_\xi(x)=\sum\limit_{k: x_k \le x} P(\xi=x_k)$
£слн распределение случайной величины абсолютно непрерывно или дискретно, то говорят также, что сама случайная величина или ее функция распределения соответственно абсолютно непрерывны или дискретны.
























Пример:
$\Omega={1,2,3,4,5,6}\quad \xi(w)=w^2, \xi(w)=sinw$ 

Пусть задана случайная величина $\xi\colon \Omega \to \{y_1,\dots y_k\}, \; \Omega = \{w_1,\dots,w_n\}$.
Нам важно знать распределение случайной величины $\xi$: $P(\xi=y_i)=P(\{w_j\colon \xi(w_j)=y_i\})$.






Математическое ожидание случайной величины
$M\xi$ ?$E\xi$? $M\xi = \sum_{w \in \Omega} \xi(w)\cdot P(w)$ 

Перепишем его по-другому.
$=y_1\left(\sum_{w\colon \xi(w)=y_1} P(w)\right)+y_2\left(\sum_{w\colon \xi(w)=y_2} P(w)\right)+\dots+y_k\left(\sum_{w\colon \xi(w)=y_k} P(w)\right)=y_1P(\xi=y_1)+y_2P(\xi=y_2)+\dots y_kP(\xi=y_k)=\sum_{j=1}^{k} y_jP(\xi=y_j)$.

линейность математического ожидания: матожидание $M(c_1\xi_1+c_2\xi_2)=c_1M\xi_1+c_2M\xi_2$ l.


Дисперсия с.в. $D\xi= M(\xi-M\xi)^2=M(\xi^2-2\xi M\xi+(M\xi)^2)=M\xi^2-2M\xi M\xi+(M\xi)^2=M\xi^2-(M\xi)^2$ 















