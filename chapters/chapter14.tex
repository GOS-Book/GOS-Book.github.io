\part[Кратные интегралы и теория поля]{Кратные интегралы и теория поля}

\chapter{Формула Грина. Потенциальные векторные поля на плоскости.}
\section{Формула Грина} 

\subsection{Криволинейные интегралы: определения, основные свойства}

\subsection{Формула Грина для клетки}

Пусть в плоскости $\bbR^2$ фиксирована некоторая прямоугольная система координат $x,y$, и пусть $\Delta$ "--- некоторая открытая клетка, т.е. $\Delta = (a;b)\times(c;d)$, а $\overline{\Delta}$ "--- ее замыкание, т.е. $\overline{\Delta}=[a;b]\times[c;d]$. Через $\partial \Delta$, как обычно, обозначим границу множества $\Delta$.


\begin{lemm}
Если функции $P(x,y)$ и $Q(x,y)$ определены и непрерывны на $\overline{\Delta}$, а их производные $P'_y$ и $Q'_x$ непрерывны и интегрируемы на $\Delta$, то справедлива формула
\begin{equation} \label{Grin}
\iint\limits_{\Delta} \left( \pd{Q}{x}-\frac{\partial P}{\partial y} \right)\,dx\,dy = \int\limits_{\partial \Delta} P\,dx + Q\,dy
\end{equation}
где обход $\partial\Delta$ совершается против часовой стрелки, если система координат $x,y$ правая, и по часовой стрелке, если система координат $x,y$ левая. %(рис 14.1 после слова правая) (рис 14.2)
\end{lemm}

\begin{proof}
По формуле сведения кратного интеграла к повторному получаем
$$
\iint\limits_{\Delta} \pd{Q}{x} \,dx\,dy = \int_{c}^{d}\limits dy \int_a^b\limits \pd{Q}{x}\,dx = \int_c^d\limits Q(b,y)\,dy - \int_c^d\limits Q(a,y)\,dy
$$
В правой части стоит разность двух интегралов: первый интеграл "--- это криволинейный интеграл по отрезку $BC$, а второй "--- интеграл отрезку $AD$. Следовательно,
$$
\iint\limits_\Delta\pd{Q}{x}\,dx\,dy = \int_{BC}\limits Q\,dy - \int_{AD}\limits Q\,dy = \int_{BC}\limits Q\,dy + \int_{DA}\limits Q\,dy.
$$

А так как интегралы от $Q\,dy$ по $AB$ и $CD$ равны нулю, то
$$
\iint\limits_\Delta\pd{Q}{x}\,dx\,dy = \int\limits_{ABCDA} Q\,dy
$$

Аналогично доказывается, что
$$
\iint\limits_\Delta\pd{P}{y}\,dx\,dy = -\int\limits_{ABCDA} P\,dx.  
$$

Лемма доказана.
\end{proof}
Формула \eqref{Grin} называется \textit{формулой Грина\rindex{формула!Грина}}. Таким образом, \textit{формула Грина справедлива для любого прямоугольника со сторонами, параллельными осям координат}.


\section{Потенциальные векторные поля на плоскости}

Если на множестве $G \subset \bbR^n$ задана векторная функция $a(M)$, то говорят, что на $G$ задано \textit{векторное поле} $a(M)$ 

В этом параграфе будем считать, что $G$ "--- это $2$-мерная область, т.е. открытое связное множество точек на плоскости.

\begin{defn}
Векторное поле $a(M)$, $M\in G$, называется \textit{потенциальным в области G}, если существует скалярная функция $u(M), M\in G$, такая, что 
$$
a(M) = \grad u(M), M \in G.
$$

В этом случае функция $u(M)$ называется \textit{потенциалом векторного поля} $a(M)$.
\end{defn}