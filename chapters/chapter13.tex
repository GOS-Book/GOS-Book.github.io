\chapter{Степенные ряды. Радиус сходимости. Бесконечная дифференцируемость суммы степенного ряда. Ряд Тейлора.}
\section{Степенные ряды. Радиус сходимости}
\subsubsection{Радиус и круг сходимости степенного ряда. Радиус сходимости}
Рассматриваем функциональные ряды вида
\begin{equation}\label{yaa11p3e1}
\sum\limits_{n=0}^{\infty} a_n(z-z_0)^n,
\end{equation}
где $z_0$ и $a_0,a_1,...,a_n,...$ "--- заданные комплексные числа, а $z$ "--- переменная, принимающая комплексные значения, т.е $z\in\bbC$, где $\bbC$ "--- множество комплексных чисел. Такие ряды называются \textit{степенными рядами}. Числа $a_0,a_1,...,a_n,...$ называются \textit{коэффициентами степенного ряда} \eqref{yaa11p3e1}.

Отметим, что у степенного ряда счет членов ведется не с единицы, а с нуля: первый член называется нулевым, второй "--- первым и т.д.

\begin{thm}
Если степенной ряд \eqref{yaa11p3e1} сходится в точке $z_1\ne z_0$, то он сходится абсолютно в любой точке $z$ из круга $|z-z_0|<r_1$, где $r_1=|z_1-z_0|$.
\end{thm}

\begin{proof}
Так как ряд \eqref{yaa11p3e1} сходится в точке $z_1$, то последовательность $\{a_n r_1^n\}$ ограничена. Пусть
$$
|a_nr_1^n|\le M \quad \forall n
$$
Положим $q=\frac{1}{r_1}|z-z_0|$. Тогда
$$
|a_n(z-z_0)^n|=|a_n|r_1^nq^n\le Mq^n\quad \forall n.
$$
Отсюда по признаку сравнения следует, что если $|z-z_0|<r_1$, то ряд \eqref{yaa11p3e1} сходится абсолютно. Теорема 1 доказана.
\end{proof}

\begin{cons}
Если ряд \eqref{yaa11p3e1} расходится в точке $z_2$, то он расходится в любой точке $z$, для которой $|z-z_0|>r_2$, где $r_2=|z_2-z_0|$.
\end{cons}

Из теоремы Абеля следует, что для степенного ряда~\eqref{yaa11p3e1} возможны три ситуации:
\begin{enumerate}
\item ряд \eqref{yaa11p3e1} сходится только в точке $z_0$;
\item ряд \eqref{yaa11p3e1} сходится во всех точках $z\in\bbC$;
\item существует число $R>0$ такое, что для всех $z$ из круга $|z-z_0|<R$ ряд сходится, а для всех $z$, для которых $|z-z_0|>R$, ряд расходится.
\end{enumerate}

\begin{defn}
Число $R>0$, обладающее свойством: ряд \eqref{yaa11p3e1} сходится, если $|z-z_0|<R$, и расходится, если $|z-z_0|>R$, называется \textit{радиусом сходимости}, а круг $|z-z_0|<R$ "--- \textit{кругом сходимости} степенного ряда~\eqref{yaa11p3e1}.

Если ряд~\eqref{yaa11p3e1} сходится только в точке $z_0$, то, по определению $R=0$. Если ряд \eqref{yaa11p3e1} сходится при любом $z\in\bbC$, то $R=+\infty$.
\end{defn}

\begin{thm}
У любого степенного ряда~\eqref{yaa11p3e1} существует радиус сходимости $R$, причем
\begin{equation}\label{yaa11p3e2}
R=\frac{1}{\overline{\lim\limits_{n\to\infty}}\sqrt[n]{|a_n|}}.
\end{equation} 
\end{thm}

\begin{proof}
Пусть $\overline{\lim\limits_{n\to\infty}}\sqrt[n]{|a_n|}=q$, и пусть сначала $0<q<+\infty$. Тогда
$$
\overline{\lim\limits_{n\to\infty}}\sqrt[n]{|a_n(z-z_0)^n|}=q\cdot |z-z_0|.
$$
Отсюда по признаку Коши для числового ряда следует, что ряд \eqref{yaa11p3e1} сходится, если $q|z-z_0|<1$, и расходится, если $q|z-z_0|>1$.

Следовательно, $R=1/q$, т.е. справедлива формула~\eqref{yaa11p3e2}.

Если $q=0$, то ряд~\eqref{yaa11p3e1} сходится при любом $z$, и поэтому $R=+\infty$. Если же $q=+\infty$, то ряд \eqref{yaa11p3e1} расходится при любом $z\ne z_0$, и поэтому $R=0$. Можно считать, что в этих случаях тоже справедлива формула~\eqref{yaa11p3e2}. 

Теорема доказана.
\end{proof}

Формула~\eqref{yaa11p3e2} называется \textit{формулой Коши"--~Адамара}\rindex{формула!Коши"---Адамара} для радиуса сходимости степенного ряда~\eqref{yaa11p3e1}.

\begin{thm} \label{yaa11p3thm3}
Степенные ряды
\begin{equation}\label{yaa11p3e5}
\sum\limits_{n=1}^{\infty} na_n(z-z_0)^{n-1},
\end{equation}

\begin{equation}\label{yaa11p3e6}
\sum\limits_{n=0}^{\infty}\frac{a_n}{n+1}(z-z_0)^{n+1}
\end{equation}
имеют тот же радиус сходимости, что и ряд \eqref{yaa11p3e1}.
\end{thm}

\begin{proof}
Заметим, что ряд \eqref{yaa11p3e5} сходится тогда и только тогда, когда сходится ряд
$$
\sum\limits_{n=1}^{\infty}na_n(z-z_0)^n.
$$
Так как $\lim\limits_{n\to\infty} \sqrt[n]{n} = 1$, то последовательности $\{\sqrt[n]{|a_n|}\}$ и $\{\sqrt[n]{n|a_n|}\}$ имеют одни и те же частные пределы. Следовательно,
$$
\overline{\lim\limits_{n\to\infty}}\sqrt[n]{n|a_n|}=\overline{\lim\limits_{n\to\infty}}\sqrt[n]{|a_n|},
$$
что, в силу формулы Коши"--~Адамара, и доказывает равенство радиусов сходимости степенных рядов~\eqref{yaa11p3e1} и \eqref{yaa11p3e5}.

Аналогично ряд~\eqref{yaa11p3e6} сходится или расходится одновременно с рядом
$$
\sum\limits_{n=0}^{\infty}\frac{a_n}{n+1}(z-z_0)^n.
$$
А так как 
$$
\overline{\lim\limits_{n\to\infty}}\sqrt[n]{\frac{|a_n|}{n+1}}=\overline{\lim\limits_{n\to\infty}}\sqrt[n]{|a_n|},
$$
то ряд~\eqref{yaa11p3e6} имеет тот же радиус сходимости, что и ряд \eqref{yaa11p3e1}. Теорема доказана.

Отметим, что ряд~\eqref{yaa11p3e5} получается из ряда~\eqref{yaa11p3e1} почленным дифференцированием, а ряд~\eqref{yaa11p3e6} "--- почленным интегрированием по отрезку~$[z_0;z]$, соединяющему точки $z_0$ и $z$. Следовательно, теорему \eqref{yaa11p3thm3} можно сформулировать так:
\textit{Исходный ряд и ряды полученные из него почленным дифференцированием и почленным интегрированием, имеют один и тот же радиус сходимости.}
\end{proof}

\begin{thm}
Пусть $R$ "--- радиус сходимости степенного ряда \eqref{yaa11p3e1}. Тогда, если $R>0$, то ряд \eqref{yaa11p3e1} сходится равномерно на любом замкнутом круге $|z_0-z|\le r$, у которого $r<R$.
\end{thm}

\begin{proof}
Из определения радиуса сходимости и теоремы Абеля следует, что ряд~\eqref{yaa11p3e1} в точке $z_1=z_0+r$ сходится абсолютно, т.е. сходится числовой ряд $\sum\limits_{n=0}^{\infty} |a_n|r^n$. А так как
$$
|a_n(z-z_0)^n|\le |a_n|r^n \quad \forall n
$$
при условии $|z-z_0|\le r$, то, согласно признаку Вейерштрасса, ряд~\eqref{yaa11p3e1} сходится равномерно в круге $|z-z_0|\le r$. Теорема доказана.
\end{proof}

\begin{cons}
Сумма степенного ряда непрерывна в круге сходимости.
\end{cons}

\begin{proof}
Пусть $R$ "--- радиус сходимости ряда~\eqref{yaa11p3e1}. В случае $R=0$ круг сходимости, согласно определению, является пустым множеством.

Пусть $R>0$. Рассмотрим некоторую точку $z_1$ из круга сходимости и положим $\delta=R-|z_1-z_0|$.

Тогда $O_{\delta}(z_1)$ содержится в $O_R(z_0)$, а $O_{\delta /2}(z_1)$ содержится в $O_r(z_0),\; r=R-\delta /2$, где ряд сходится равномерно. А так как члены ряда~\eqref{yaa11p3e1} непрерывны в точке $z_1$, то его сумма тоже непрерывна в точке $z_1$.
 
Следствие доказано.
\end{proof}

Рассмотрим теперь случай, когда степенной ряд сходится в некоторой точке, лежащей на границе его круга сходимости. Рассмотрим ряды вида

\begin{equation}\label{yaa11p32e2}
\sum\limits_{n=0}^{\infty} a_nz^n.
\end{equation}

\begin{thm}
Если $R$ "--- радиус сходимости ряда~\eqref{yaa11p32e2}, и ряд сходится при $z=R$, то он сходится равномерно на отрезке $[0;R]$ действительной оси.
\end{thm}

\begin{proof}
Если $z\in[0;R]$, то
$$
a_nx^n=a_nR^n\left(\frac{x}{R}\right)^n.
$$
Ряд $\sum\limits_{n=0}^{\infty} a_nR^n$ сходится равномерно относительно $x$, а последовательность
$$
b_n=\left(\frac{x}{R}\right)^n,\quad n\in \bbN,
$$
при любом $x$ монотонна и на $[0;R]$ ограничена:
$$
0\le \left(\frac{x}{R}\right)^n\le 1
$$
Поэтому, согласно принципу Абеля, ряд~\eqref{yaa11p32e2} сходится равномерно на $[0;R]$. 

Теорема доказана.
\end{proof}

Эта теорема называется \textit{Второй теоремой Абеля}.


\section{Бесконечная дифференцируемость суммы степенного ряда}
Рассмотрим степенной ряд
\begin{equation} \label{ch13.3eq1}
\sum\limits_{n = 0}^{\infty} a_n (x - x_0)^n
\end{equation}
\begin{thm} \label{ch13.3thm2}
Пусть $R$ "--- радиус сходимости степенного ряда $(\ref{ch13.3eq1})$, а $f(x)$ "--- сумма этого ряда. Тогда, если $R > 0$, то функция $f(x)$ в интервале $(x_0 - R, x_0 + R)$ имеет производные любого порядка, которые находятся почленным дифференцированием ряда $(\ref{ch13.3eq1})$.
\end{thm}

\begin{proof}
Действительно, согласно теореме $\ref{yaa11p3thm3}$, почленно продифференцированный ряд имеет тот же радиус сходимости $R$, что и ряд $(\ref{ch13.3eq1})$, поэтому он сходится равномерно на любом отрезке $[x_0 - r, x_0 + r] \subset (x_0 - R, x_0 + R)$, и утверждение теоремы следует из теоремы о почленном дифференцировании функционального ряда.
\end{proof}

\begin{thm} \label{ch13.3thm3}
Если функция $f(x)$ является аналитической в точке $x_0$, т.е. в некоторой $\delta$-окрестности точки $x_0$ она является суммой степенного ряда вида $(\ref{ch13.3eq1})$, то
\begin{equation} \label{ch13.3eq2}
a_n = \frac{1}{n!} f^{(n)}(x_0), \quad n = 0, 1, 2, \ldots
\end{equation}
\end{thm}
\begin{proof}
Действительно, продифференцировав $n$ раз обе части равенства
$$
f(x) = \sum\limits_{n = 0}^{\infty} a_n(x - x_0)^n
$$
и положив $x = x_0$, получим $f^{(n)}(x_0) = n!\cdot a_n$.
\end{proof}
\section{Ряд Тейлора}

\begin{defn}
Пусть функция $f(x)$ определена и бесконечно дифференцируема в некоторой окрестности точки $x_0$. Тогда степенной ряд $(\ref{ch13.3eq1})$, коэффициенты которого определены по формулам $(\ref{ch13.3eq2})$, называется \textit{рядом Тейлора\rindex{ряд!Тейлора} функции $f(x)$ в точке $x_0$}.

В случае, когда $x_0 = 0$, ряд Тейлора называют \textit{рядом Маклорена}\rindex{ряд!Маклорена}.

Из теорем $\ref{ch13.3thm2}$ и $\ref{ch13.3thm3}$ следует, что если функция $f(x)$ аналитическая в точке $x_0$, то она в некоторой окрестности точки $x_0$ бесконечно дифференцируема и
$$
f(x) = \sum\limits_{n = 0}^{\infty} \frac{1}{n!}f^{(n)}(x_0)(x - x_0)^n,
$$
т.е. в этой окрестности она разлагается в ряд Тейлора по степеням $x - x_0$.
\end{defn}

Рассмотрим формулу Тейлора для функции $f(x)$, которая определена и бесконечно дифференцируема в окрестности точки $x_0$
\begin{equation} \label{ch13.4eq1}
f(x) = \sum\limits_{k = 0}^{n} \frac{1}{k!}f^{(k)}(x_0)(x - x_0)^k + r_n(x),
\end{equation}
где $r_n(x)$ "--- $n$-й остаточный член формулы Тейлора. Из нее видно, что \textit{функция $f(x)$ есть сумма своего ряда Тейлора в некоторой $\delta$-окрестности точки $x_0$ тогда и только тогда, когда}
$$
\lim\limits_{n \to \infty} r_n(x) = 0 \quad \forall x \in O_\delta(x_0).
$$
\begin{thm}
Пусть функция $f(x)$ определена и бесконечно дифференцируема в некоторой $\delta$-окрестности точки $x_0$. Тогда, если
$$
\exists M\cquad |f^{(n)}(x)| \le M \quad \forall x \in O_\delta(x_0), \quad \forall n,
$$ 
то
$$
f(x) = \sum\limits_{n = 0}^{\infty} \frac{f^{(n)}(x_0)}{n!}(x - x_0)^n \quad \forall x \in O_\delta(x_0).
$$
\end{thm}
\begin{proof}
Для функции $f(x)$ напишем формулу Тейлора $(\ref{ch13.4eq1})$ с остаточным членом в форме Лагранжа:
$$
r_n(x) = \frac{f^{(n+1)}(\xi)}{(n + 1)!} (x - x_0)^{n + 1},
$$
где $\xi$ лежит между $x$ и $x_0$. Тогда $|f^{(n+1)}(\xi)| \le M$, и поэтому
$$
|r_n(x)| \le M \cdot \frac{\delta^{n + 1}}{(n + 1)!}
$$
для любого $n$ и любого $x \in O_\delta(x_0)$. А это и доказывает теорему, так как
\begin{equation*}
\lim\limits_{n \to \infty} \frac{\delta^{n + 1}}{(n + 1)!} = 0. \qedhere
\end{equation*}
\end{proof}
