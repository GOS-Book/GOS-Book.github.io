\chapter{Непрерывность преобразования Фурье абсолютно интегрируемой функции. Преобразование Фурье производной и производная преобразования Фурье.}
\section{Непрерывность преобразования Фурье абсолютно интегрируемой функции}
\section{Преобразование Фурье производной и производная преобразования Фурье}

Функцию $f$, определенную на $\bbR$, будем называть \textit{кусочно непрерывной}, если она кусочно непрерывна на любом конечном интервале. Если же она на любом конечном интервале кусочно дифференцируема, то будем говорить, что она \textit{кусочно дифференцируема на $\bbR$}. Аналогично определяются и \textit{кусочно непрерывно дифференцируемые на $\bbR$} функции. 

Заметим, что если функция $f$ непрерывна и кусочно непрерывно дифференцируема, то она является обобщенной первообразной для производноцй $f'$.

\begin{lemm}
Пусть функция $f(x)$ непрерывна и кусочно непрерывно дифференцируема на $bbR$. Тогда, если $f(x)$ и $f'(x)$ абсолютно интегрируемы на $bbR$, то $f(x)\to 0$ при $x\to\pm\infty$
\end{lemm}

Действительно, из равенства
$$
f(x)=f(c)+\int\limits_c^x f'(x)dt,\quad c\in\bbR,
$$
и сходимости интеграла $f'(x)$ на $\bbR$ следует, что пределы у $f(x)$ при $x\to\pm\infty$ существуют, а из сходимости интеграла от $f(x)$ на $\bbR$ следует, что эти пределы равны нулю.

\begin{thm}
Пусть функция $f(x)$ непрерывна и кусочно непрерывно дифференцируема на $\bbR$. Тогда, если $f(x)$ и $f'(x)$ абсолютно интегрируемы на $\bbR$, то
$$
F[f']=i\xi\widehat{f}(\xi),\quad F^{-1}[f']=-i\xi \widetilde{f}(\xi).
$$
\end{thm}

\begin{proof}
По формуле интегрирования по частям получаем
$$
F[f']=\frac{1}{\sqrt{2\pi}}\int\limits_{-\infty}^{+\infty} f'(x)e^{-i\xi x}dx=\frac{1}{\sqrt{2\pi}}f(x)e^{-i\xi x}	
\bigr|^{-\infty}_{+\infty} - \frac{1}{\sqrt{2\pi}}\int\limits_{-\infty}^{+\infty}f(x)(-i\xi)e^{-i\xi x}dx.
$$
В силу леммы, внеинтегральные члены равны нулю, поэтому
$$
F[f']=i\xi\widetilde{f}(\xi).
$$
Аналогично доказывается и вторая формула.
\end{proof}

\begin{cons}
Если $f,f',...,f^{(n)}$ непрерывны и абсолютно интегрируемы на $\bbR$, то
$$
F[f^{(k)}]=(i\xi)^kF[f],
F^{-1}[g^{(k)}]=(-i\xi)^kF^{-1}[f], \quad k=0,1,...,n.
$$
В частности
$$
\widehat{f}(\xi)=o\left(\frac{1}{\xi^n}\right),\quad \widetilde{f}(\xi)=o\left(\frac{1}{\xi^n}\right)
$$
при $\xi\to\pm\infty$.
\end{cons}

\begin{thm}
Если функции $f(x)$ и  $xf(x)$ абсолютно интегрируемы на $\bbR$, то $\widehat{f}(\xi)$ и $\widetilde{f}(\xi)$ непрерывно дифференцируемы на $\bbR$ и 
$$
\frac{d\widehat{f}}{d\xi}=-iF[x[f(x)],\quad \frac{d\widetilde{f}}{d\xi}=iF^{-1}[x[f(x)].
$$
\end{thm}

\begin{proof}
По признаку Вейерштрасса интегралы
$$
\int\limits\_{-\infty}^{+\infty}f(x)e^{-i\xi x}dx\quad \text{и} -i\int\limits\_{-\infty}^{+\infty}xf(x)e^{-i\xi x}dx
$$
сходятся равномерно по $\xi$ на $\bbR$, поэтому они непрерывны и производная по $\xi$ от первого из них равна второму. Следовательно,
$$
\frac{d\widehat{f}}{\xi}=-iF[xf(x)].
$$

Аналогично доказывается и вторая формула.
\end{proof}

\begin{cons}
Если функции $f(x),xf(x),...,x^nf(x)$ абсолютно интегрируемы на $\bbR$, то $\widehat{f}(\xi)$ и $\widetilde{f}(\xi)$ $n$ раз непрерывно дифференцируемы на $\bbR$ и 
$$
\frac{d^k\widehat{f}}{d\xi^k}=(-i)^kF[x^kf(x)],\quad \frac{d^k\widetilde{f}}{d\xi^k} = i^kF^{-1}[x^kf(x)],\quad k=0,1,...,n.
$$
\end{cons}