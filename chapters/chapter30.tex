\chapter{Полная система событий. Формула полной вероятности. Формула Байеса.}
\section{Полная система событий}

\subsection{Классическое определение вероятности}

Рассмотрим обычную игральную кость --- кубик, на каждой из шести граней которого нанесены числа от 1 до 6. У нас есть всего 6 вариантов того, как этот кубик может упасть на стол после случайного бросания Какая же цифра выпадет на кубике? 
\begin{center}
1,2,3,4,5,6 --- всего 6 исходов нашего испытания.
\end{center}
Каким свойствами обладают эти варианты?
\begin{enumerate}
\item Хотя бы один из исходов обязательно случится. (Исходы образуют \textit{полную систему событий})
\item Исходы попарно несовместны. (Никакие два одновременно не происходят.)
\item Исходы равновероятны.
\end{enumerate}

Подобных примеров можно придумать великое множество. Например, пусть есть игральная колода из 36 карт. А исход --- взаимное расположение карт друг за другом после тщательной перетасовки (их 36! всего способов переставить карты внутри множества от 1 до 36). Такая система событий тоже обладает перечисленными свойствами.

Пусть в рамках какого-то эксперимента есть какие-то исходы $w_1, \dots,w_n$ (их конечное количество), и они обладают ровно этими тремя перечисленными выше свойствами. Тогда эти исходы называют \textit{элементарными}. Тогда по определению считают
$$
\bbP(w_i)=\frac{1}{n}.
$$
где $\bbP(w_i)$ --- вероятность произвольного, элементарного исхода $w_i$. Это очень естественно. Действительно, если хотя бы один из этих исходов произойдет, причем на самом деле ровно один и эти исходы равновозможны, то в самом банальном смысле этого слово, есственно сказать, что вероятность каждого их этих исходов --- это $1/n$. Тогда в задаче про кубик $\bbP(w_i) =frac{1}{6}$, а в задаче про карты $\bbP(w_i) = \frac{1}{36}$.

Но как мы понимаем элементарные события --- лишь кирпичики, из которых состоит наша вероятностная вселенная. Могут происходить более сложные, чем наши элементарные, события. Например, в качестве события в задаче про игральную кость можно рассмотреть $A$ --- игральная кость выпала четной стороной кверху. Это, очевидно, означает, что она выпала либо стороной 2, либо стороной 4, либо стороной 6.
\section{Формула полной вероятности}
\section{Формула Байеса}